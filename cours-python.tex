\documentclass[10pt]{book}
\usepackage[utf8]{inputenc}
\usepackage[T1]{fontenc}
\usepackage[francais]{babel}
\usepackage{lmodern}
\usepackage{amsmath}
\usepackage{mathptmx}
\usepackage{graphicx}
\usepackage{geometry}
\geometry{
  margin=1.5cm,
  footskip=1.5cm,
}

\let\cleardoublepage\clearpage

% define PDF format options
\usepackage{hyperref}
\hypersetup{ % configuration de hyperref
  draft=false,          		% true pour imprimer sans les liens
  bookmarks=true,       		% Signets
  bookmarksnumbered=true,   % Signets numérotés
  pdfpagemode=None,         % Signets/vignettes fermé à l'ouverture
  pdfstartview=FitH,        % La page prend toute la largeur
  pdfpagelayout=SinglePage, % Vue par page
  colorlinks=true,     			% Liens en couleur
  urlcolor=blue,     				% Couleur des liens externes
  pdfborder={0 0 0},        % Style de bordure : ici, pas de bordure
  pdfauthor={Patrick Fuchs et Pierre Poulain},
  pdftitle={Cours de Python},
  pdfsubject={Cours de Python},
  pdfkeywords={programmation,Python,biologie},
  pdfcreator={pandoc},
  pdfproducer={pandoc}
}

\usepackage{upquote}
\usepackage{longtable,booktabs}


%====================================================================
% Conversion des liens web en note de bas de page
%====================================================================
\renewcommand{\href}[2]{#2\footnote{\url{#1}}}


%==============================================================================
% load listing package
% and redefine the vebatim environment
%==============================================================================
\usepackage{listings}
\usepackage{color}
\definecolor{gray}{rgb}{0.7,0.7,0.7}
\let\verbatim\relax
\lstnewenvironment{verbatim}[1][]{
    \lstset{
        basicstyle=\linespread{0.8}\footnotesize\ttfamily,
        literate=
        {á}{{\'a}}1 {é}{{\'e}}1 {í}{{\'i}}1 {ó}{{\'o}}1 {ú}{{\'u}}1
        {Á}{{\'A}}1 {É}{{\'E}}1 {Í}{{\'I}}1 {Ó}{{\'O}}1 {Ú}{{\'U}}1
        {à}{{\`a}}1 {è}{{\`e}}1 {ì}{{\`i}}1 {ò}{{\`o}}1 {ù}{{\`u}}1
        {À}{{\`A}}1 {È}{{\'E}}1 {Ì}{{\`I}}1 {Ò}{{\`O}}1 {Ù}{{\`U}}1
        {ä}{{\"a}}1 {ë}{{\"e}}1 {ï}{{\"i}}1 {ö}{{\"o}}1 {ü}{{\"u}}1
        {Ä}{{\"A}}1 {Ë}{{\"E}}1 {Ï}{{\"I}}1 {Ö}{{\"O}}1 {Ü}{{\"U}}1
        {â}{{\^a}}1 {ê}{{\^e}}1 {î}{{\^i}}1 {ô}{{\^o}}1 {û}{{\^u}}1
        {Â}{{\^A}}1 {Ê}{{\^E}}1 {Î}{{\^I}}1 {Ô}{{\^O}}1 {Û}{{\^U}}1
        {œ}{{\oe}}1 {Œ}{{\OE}}1 {æ}{{\ae}}1 {Æ}{{\AE}}1 {ß}{{\ss}}1
        {ű}{{\H{u}}}1 {Ű}{{\H{U}}}1 {ő}{{\H{o}}}1 {Ő}{{\H{O}}}1
        {ç}{{\c c}}1 {Ç}{{\c C}}1 {ø}{{\o}}1 {å}{{\r a}}1 {Å}{{\r A}}1
        {€}{{\euro}}1 {£}{{\pounds}}1 {«}{{\guillemotleft}}1
        {»}{{\guillemotright}}1 {ñ}{{\~n}}1 {Ñ}{{\~N}}1 {¿}{{?`}}1 ,
        tabsize=4,
        columns=fixed,
        numbers=left,
        stepnumber=1,
        numberstyle=\tiny\color{gray},
        frame=leftline,
        framerule=0.1pt,
        rulecolor=\color{gray},
        frameround=tttt,
        framesep=3pt,  % space between code and frame
        numbersep=5pt, % space between numbers and code
        xleftmargin=10pt, % left margin
        xrightmargin=-10pt, % right margin
        #1
    }
}{}

%==============================================================================
% define page header and footer
%==============================================================================
\usepackage{fancyhdr}
\pagestyle{fancy}
\fancyhf{} % clear all headers and footer fields
\fancyhead[LE,RO]{\nouppercase{\leftmark}}
\fancyhead[RE,LO]{\nouppercase{\rightmark}}
\fancyfoot[LE,RO]{\thepage}
\fancyfoot[RE,LO]{\it Cours de Python / Université Paris Diderot - Paris 7 / UFR Sciences du Vivant}



%==============================================================================
% boxes
%==============================================================================
\usepackage{framed}

\newenvironment{box-rem}%
{\vspace{5mm}\noindent\textbf{Remarque}\ \hrulefill}%
{\vspace*{-2mm}\noindent\hrulefill\vspace{5mm}}

\newenvironment{box-adv}%
{\vspace{5mm}\noindent\textbf{Conseil}\ \hrulefill}%
{\vspace*{-2mm}\noindent\hrulefill\vspace{5mm}}

\newenvironment{box-warn}%
{\vspace{5mm}\noindent\textbf{Attention}\ \hrulefill}%
{\vspace*{-2mm}\noindent\hrulefill\vspace{5mm}}

\newenvironment{box-def}%
{\vspace{5mm}\noindent\textbf{Définition}\ \hrulefill}%
{\vspace*{-2mm}\noindent\hrulefill\vspace{5mm}}

\newenvironment{box-more}%
{\vspace{5mm}\noindent\textbf{Pour aller plus loin}\ \hrulefill}%
{\vspace*{-2mm}\noindent\hrulefill\vspace{5mm}}

%==============================================================================
% start document
%==============================================================================
\begin{document}

%==============================================================================
% title page
%==============================================================================
\thispagestyle{empty}

\begin{titlepage}

\begin{center}

\includegraphics[width=10cm]{img/LogoUPD_USPC.png}

\vspace{3cm}

{\Huge \bf Cours de Python}\\
\includegraphics[width=5cm]{img/logo_python.png} \\
\verb@https://python.sdv.univ-paris-diderot.fr/@
\vspace{2cm}

{\large
	{\bf Patrick Fuchs} et {\bf Pierre Poulain} \\
	{\tt prénom [point] nom [arobase] univ-paris-diderot [point] fr}
}

\vspace{3 cm}

version du \today

\vspace{3cm}
Université Paris Diderot-Paris 7, Paris, France

\vfill

\begin{minipage}{0.80\textwidth}
\footnotesize
Ce document est sous licence \\
Creative Commons Attribution - Partage dans les Mêmes Conditions 3.0 France \\
(CC BY-SA 3.0 FR) \\
\url{https://creativecommons.org/licenses/by-sa/3.0/fr/}
\end{minipage}
\begin{minipage}{0.15\textwidth}
\includegraphics{img/logo_CC-BY-SA.png}
\end{minipage}

\end{center}
\end{titlepage}

%==============================================================================
% define margins
%==============================================================================
\newgeometry{left=2cm,right=2cm,top=2cm,bottom=2.5cm}

%==============================================================================
% add table of content
%==============================================================================
\setcounter{tocdepth}{1}
\tableofcontents

\clearpage
\include{tex/00_avant_propos}
\include{tex/01_introduction}
\include{tex/02_variables}
\include{tex/03_affichage}
\include{tex/04_listes}
\include{tex/05_boucles_comparaisons}
\include{tex/06_tests}
\include{tex/07_fichiers}
\include{tex/08_modules}
\include{tex/09_fonctions}
\include{tex/10_plus_sur_les_chaines_de_caracteres}
\include{tex/11_plus_sur_les_listes}
\include{tex/12_plus_sur_les_fonctions}
\include{tex/13_dictionnaires_tuples}
\include{tex/14_creation_modules}
\include{tex/15_bonnes_pratiques}
\include{tex/16_expressions_regulieres}
\include{tex/17_modules_interet_bioinfo}
\include{tex/18_jupyter}
\include{tex/19_avoir_la_classe_avec_les_objets}
\include{tex/20_tkinter}
\include{tex/21_remarques_complementaires}
\include{tex/21_remarques_complementaires_vrac}
\include{tex/22_mini_projets}

\appendix

\include{tex/annexe_formats_fichiers}
\include{tex/annexe_install_python}

\end{document}
